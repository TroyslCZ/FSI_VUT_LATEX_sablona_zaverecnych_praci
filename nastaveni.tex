%%%%%%%%%%%%%%%%%%%%%%%%%%%%%%%%%%%%%%%%%%%%%%%%%%%%%%%%%%%%%%%%%%%%%%%%
%% ŠABLONA ZÁVĚREČNÝCH PRACÍ FSI na VUT                               %%
%%                                                                    %%
%% Autor: Antonín Sochor, antonin.sochor@vutbr.cz			     %%                            
%% Verze: 1.0 (26. 05. 2023)							     %%
%%                                                                    %%
%% Šablona vznikla jako bakalářská práce v akademickém roce 2022/2023 %%
%% a je možné ji dále modifikovat a libovolně využívat.		     %%
%%											     %%
%% ------------------------------------------------------------------ %%
%%											     %%
%% NASTAVENÍ									     %%
%%											     %%
%% V tomto souboru se nastavují veškeré informace a parametry pro     %%
%% šablonu závěrečné práce. Nehodící se parametry je možné smazat     %%
%% nebo zakomentovat pomocí symbolu "%".						%%
%%%%%%%%%%%%%%%%%%%%%%%%%%%%%%%%%%%%%%%%%%%%%%%%%%%%%%%%%%%%%%%%%%%%%%%%



%------------------------------------%
%  ZÁKLADNÍ ÚDAJE O ZÁVĚREČNÉ PRÁCI  %
%------------------------------------%

%% DRUH ZÁVĚREČNÉ PRÁCE
%  Typ závěrečné práce se zadává ve formě zkratky: 
%           - "bc" = bakalářská práce
%           - "ing" = diplomová práce
%           - "phd" = dizertační práce
\prace{bc}
%\prace{ing}
%\prace{phd}

%% NÁZEV ZÁVĚREČNÉ PRÁCE
%  Název se zadává celý, je možno jej zkopírovat ze Studisu nebo
%  z úvodní strany závěrečné práce.
\nazev{Realizace šablony závěrečných prací v LATEX}

%% ODKAZ NA ELEKTRONICKOU PUBLIKACI ZÁVĚREČNÉ PRÁCE
%  Odkaz se zkopíruje ze studisu, nezkracuje se.
\odkaz{https://www.vut.cz/studenti/zav-prace/detail/145841}

%% AUTOR ZÁVĚREČNÉ PRÁCE
%  Jméno se zadává ve tvaru {tituly před}{jméno}{příjmení}{tituly za},
%  přičemž jednotlivé tituly (pokud jich je více) se oddělují čárkou
%  a pevnou/nezlomitelnou mezerou (",~").
%  Pozn.: Pokud autor práce nemá žádné tituly, příslušné kolonky se
%         ponechají prázdné, tj. {}  
\autor{}{Antonín}{Sochor}{}

%% VEDOUCÍ ZÁVĚREČNÉ PRÁCE
%  Jméno se zadává ve tvaru {tituly před}{jméno}{příjmení}{tituly za},
%  přičemž jednotlivé tituly (pokud jich je více) se oddělují čárkou
%  a pevnou/nezlomitelnou mezerou (",~").
%  Pozn.: Pokud vedoucí práce nemá žádné tituly, příslušné kolonky se
%         ponechají prázdné, tj. {}  
\vedouci{Ing.}{Kamil}{Staněk}{}

%% ÚSTAV
%  Ústav se zadává celým názvem. Ze seznamu níže stačí vybrat ústav,
%  na kterém autor práce studuje, případně ten, na kterém proběhne
%  SZZ a obhajoba závěrečné práce. Zbylé ústavy je možné smazat nebo
%  zakomentovat pomocí "%".
%\ustav{Ústav matematiky}
%\ustav{Ústav fyzikálního inženýrství}
%\ustav{Ústav mechaniky těles, mechatroniky a biomechaniky}
%\ustav{Ústav materiálových věd a inženýrství}
%\ustav{Ústav konstruování}
%\ustav{Energetický ústav}
%\ustav{Ústav strojírenské technologie}
%\ustav{Ústav výrobních strojů, systémů a robotiky}
%\ustav{Ústav procesního inženýrství}
%\ustav{Ústav automobilního a dopravního inženýrsví}
%\ustav{Letecký ústav}
\ustav{Ústav automatizace a informatiky}
%\ustav{Ústav jazyků}
	
%% MĚSTO
%  Město se zadává takové, ve kterém bude probíhat SZZ a obhajoba
%  závěrečné práce, typicky Brno.  
\mesto{Brno}
	
%% ROK OBHAJOBY
%  Rok obhajoby se zdává ve formátu RRRR (R = rok). Jedná se o rok,
%  ve kterém proběhne SZZ a obhajoba závěrečné práce.   
\obhajoba{2023}

%% DATUM ODEVZDÁNÍ
%  Datum odevzdání se zadává ve formátu DD. MM. RRRR (D = den,
%  M = měsíc, R = rok), přičemž se jedná o datum, do kdy je
%  možné do Studisu závěrečnou práci odevzdat.
\odevzdani{26. 5. 2023}

%% JAZYK ZPRACOVÁNÍ
%  Na výběr jsou čtyři možnosti:
%	- hlavní jazyk: čeština		vedlejší jazyk: angličtina,
%	- hlavní jazyk: slovenština	vedlejší jazyk: angličtina,
%	- hlavní jazyk: angličtina	vedlejší jazyk: čeština,
%	- hlavní jazyk: angličtina	vedlejší jazyk: slovenština.
%  Hlavní jazyk je ten, ve kterém bude závěrečná práce napsána, vedlejší
%  jazyk je použit na úvodní straně a jako překlad abstraktu a klíčových
%  slov.
%  Jazyk zpracování se zadává ve formě zkratky:
%	- cz-en = hlavní jazyk čeština, vedlejší jazyk angličtina,
%	- sk-en = hlavní jazyk slovenština, vedlejší jazyk slovenština,
%	- en-cz = hlavní jazyk angličtina, vedlejší jazyk čeština,
%	- en-sk = hlavní jazyk angličtina, vedlejší jazyk slovenština.
%  Pokud bude uvedena nesprávná nebo žádná zkratka, bude automaticky
%  nastavena volba hlavní jazyk čeština, vedlejší jazyk angličtina.
\jazyk{cz-en}
%\jazyk{sk-en}
%\jazyk{en-cz}
%\jazyk{en-sk}

%-------------------------------------%


%--------------------%
%  VZHLED DOKUMENTU  %
%--------------------%

%% FORMÁTOVÁNÍ ODSTAVCŮ
%  Uživatel má při formátování odstavů na výběr ze dvou možností:
%	1. Nový odstavec bude začínat odstavcovou zarážkou, tj. první
%	   slovo nového odstavce bude uskočeno o 1 cm doprava.
%	2. Nový odstavec bude bez odstavcové zarážky a od předchozího
%	   odstavce bude odsazen o 6 pt.
%  Toto uživatel volí pomocí příkazu "\odstavec{}", přičemž jako argument
%  tohoto příkazu je zadávána možnost "1" nebo "2" dle seznamu výše.
%  Pozn.: Pokud nebude vyplněna žádná nebo nesprávná možnost, bude
%	    automaticky nastavena možnost č. 2.
\odstavec{2}

%% ČÍSLOVÁNÍ OBRÁZKŮ, TABULEK A ROVNIC
%  Uživatel má při číslování na výběr ze dvou možností:
%  	1. Číslo obrázku, tabulky a rovnice obsahuje i číslo hlavní
%	   kapitoly, tj. číslo bude vypadat např.: Obrázek 1.3: Auto.
%	   kde první číslo je číslo hlavní kapitoly a druhé číslo je
%	   pořadové číslo obrázku v dané kapitole.
%	2. Číslo obrázku, tabulky a rovnice neobsahuje číslo kapitoly,
%	   bude pouze ve formátu: Obrázek 3: Auto.
%  Uživatel pomocí příkazu "\cislovani{}" zvolí jednu z výše uvedených
%  možností, přičemž její číslo "1" nebo "2" zadá jako argument příkazu.
%  Pozn.: Pokud nebude vyplněna žádná nebo nesprávná možnost, bude
%  	    automaticky nastavena možnost č. 2.
\cislovani{2}
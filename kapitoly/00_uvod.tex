%----------------------%
%  KAPITOLA 01 - ÚVOD  %
%----------------------%

\chapter*{ÚVOD}
\phantomsection
\addcontentsline{toc}{chapter}{ÚVOD}

\LaTeX je nejvýznamnější a nejrozšířenější nadstavba systému \TeX. Byla vytvořena Leslie Lamportem za účelem zpřístupnění tohoto systému co nejvíce uživatelům. Díky tomu tak není nutné přemýšlet, jak tvořený text naformátovat a vysázet, ale pouze jej sepsat~a~o~překlad zdrojového textu a vysázení dokumentu se postará \LaTeX. O vysoké popularitě tohoto systému svědčí stále se rozšiřující množství balíčků, které přidávají desítky funkcionalit a vylepšení dle potřeb a požadavků uživatelů. Velkou výhodou systému \LaTeX\ je jeho univerzálnost. Jádrem celého systému je překladač jazyka \TeX, který je na všech platformách totožný. Uživatel tak má možnost využít libovolný operační systém i editor.

Jeden z rozšiřujících balíčků je i předmětem této bakalářské práce -- šablona závěrečných prací pro Fakultu strojního inženýrství na Vysokém učení technickém v Brně. Následující řádky tak slouží i jako referenční příručka k užívání této šablony. Čtenář se dozví co všechno šablona obsahuje a jak vypadají a fungují její jednotlivé části. Zároveň nalezne seznam nejpoužívanějších \LaTeX\ editorů se souhrnem jejich výhod a nevýhod a podrobnějším popisem jednoho z editorů, který byl autorem využíván pro tvorbu šablony a také této závěrečné práce.

Nedílnou součástí závěrečné práce je příloha ve formě PDF dokumentu \uv{Čti mě}, který obsahuje popis vstupního nastavení šablony, tj. kde vyplnit informace o autorovi, o závěrečné práci, jak vložit úvodní stranu, zadání závěrečné práce a další informace, které doplňují vlastní textový obsah práce.

%----------------------%
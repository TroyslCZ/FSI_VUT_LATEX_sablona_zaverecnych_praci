\chapter{Příloha č. 1 -- soubor ČTI MĚ}
Šablona závěrečné práce je tvořena následujícími soubory:

\begin{enumerate}
	\item \verb|nastaveni.tex| -- soubor pro zadání základních údajů o závěrečné práci (druh závěrečné práce, název, odkaz na elektronickou publikaci, autor, vedoucí, ústav, město, rok obhajoby, datum odevzdání a jazyk zpracování). Dále je nastavováno formátování odstavců a číslování obrázků, tabulek a rovnic.
	\item \verb|zaverecna_prace.tex| -- hlavní pracovní soubor, který se překládá překladačem Lua\LaTeX a BIB\TeX. V souboru se aktivuje samotná šablona, aktivují se volitelné externí balíčky, vkládá se titulní strana a zadání. Do tohoto souboru uživatel také přímo vepisuje úvodní textové náležitosti (abstrakt, klíčová slova, prohlášení a poděkování), vkládá se jednotlivé kapitoly a probíhá generování obsahů.
	\item \verb|sablona.sty| -- balíček, zdrojový kód šablony.
	\item \verb|citace.bib| -- soubor pro vkládání zdrojového textu použité literatury pro program BIB\TeX.
	\item \verb|czplain.bst| -- stylový balíček pro program BIB\TeX, který upravuje styl citací dle ČSN ISO 690.
	\item \verb|zaverecna_prace.pdf| -- výsledný PDF dokument celé závěrečné práce.	
\end{enumerate}

Pomocné adresáře:

\begin{enumerate}
	\item \verb|fonty| -- složka se zdrojovými soubory VUT fontů použitých v šabloně.
	\item \verb|kapitoly| -- složka pro ukládání souborů ve formátu \verb|.tex| s textovým obsahem jednotlivých kapitol. Kapitoly ÚVOD a ZÁVĚR pojmenovat jako \verb|00_uvod.tex| a \verb|00_zaver.tex|.
	\item \verb|loga| -- složka s logy fakulty a univerzity, které jsou použity v šabloně.
	\item \verb|obrázky| -- složka pro ukládání obrázků použitých v textu práce.
	\item \verb|soubory| -- složka pro titulní stranu a zadání ve formátu \verb|.pdf|. Soubory pojmenovat jako \verb|titulni_strana.pdf| a \verb|zadani.pdf|.
\end{enumerate}

Soubory s příponami \verb|.aux|, \verb|.bbl|, \verb|.blg|, \verb|.lof|, \verb|.log|, \verb|.lot|, \verb|.out| a \verb|.toc| jsou soubory vytvářené systémem \LaTeX\ během překladu zdrojového textu. Do těchto souborů si systém ukládá potřebná data používaná pro generování výsledného PDF dokumentu.
%--------------------------------------%
%  KAPITOLA 03 - ZHODNOCENÍ A DISKUZE  %
%--------------------------------------%

\chapter{ZHODNOCENÍ A DISKUZE}
Při tvorbě šablony byl kladen důraz na jednoduchost a použitelnost. Cílem autora bylo vytvořit šablonu a popsat její jednotlivé části tak, aby bylo možné pochopit její fungování i bez větších či hlubších znalostí systému \LaTeX. Veškeré vytvořené příkazy jsou proto doplněny o popisky jak s nimi pracovat, co je možné zadávat do jejich případných argumentů a jaké jsou jejich přednastavené hodnoty.

Šablona je adekvátním nástrojem pro psaní závěrečných vysokoškolských kvalifikačních prací na Fakultě strojního inženýrství Vysokého učení technického v Brně. Vzhledem k podrobným komentářům a jednotné struktuře zdrojového kódu je šablona přehlednější než některé ze využívaných šablon na VUT. Šablona je tak více otevřenější pro další rozvoj a rozšíření na další fakulty.
s
\vspace{8pt}
\textbf{Známé problémy}\\
Při vytváření nečíslovaných kapitol (úvod, závěr, literatura a seznamy) nedochází k automatickému vytváření kotvy pro křížové odkazy. To je vyřešeno pomocí vložení imaginární sekce \verb|\phantomsection|, která tuto kotvu vytvoří. Křížový odkaz ovšem chybně odkazuje na konec dané kapitoly, nikoliv na její začátek.

Kvůli tvorbě nečíslovaných kapitol a jejich ručnímu vkládání do obsahu byla zjištěna nekompatibilita s externím balíčkem \verb|amsmath|, který rozšiřuje možnosti práce s matematickými rovnicemi. Při pokusu o překlad s tímto aktivovaným balíčkem vyskočí chybová hláška \uv{\textit{LaTeX Error: Something's wrong--perhaps a missing \textbackslash item}}. Jedná se o obecnou chybovou hlášku systému \LaTeX, která přímo neukazuje na objevený problém. Při prohledávání \verb|.log| souboru se tato chybová hláška objevuje v místě vkládání nečíslovaných kapitol do obsahu. Za balíčkem stojí sdružení AMS -- American Mathematical Society, kterému byl tento problém nahlášen a lze proto předpokládat, že tato chyba bude bude opravena v některé z budoucích aktualizací balíčku.

\vspace{8pt}
\textbf{Možnosti dalšího rozvoje}\\
Dalším rozvojem (například v rámci diplomové práce) je možné šablonu rozšířit o využití i pro semestrální práce. Při zadání typu práce jako \uv{semestrální} by šablona automaticky vyřadila zpracování úvodních textových náležitostí (abstrakt, klíčová slova, bibliografická citace, čestné prohlášení a poděkování) a případně také seznamy použité literatury, obrázků, tabulek a dalších.

Již komplexnějším rozvojem se jeví možnost tvorby nového stylového balíčku pro tvorbu citací tak, aby splňoval veškeré náležitosti citační normy po její aktualizaci na konci roku 2022 a aby tento stylový balíček pří přímou součástí zdrojového kódu šablony.

%--------------------------------------%
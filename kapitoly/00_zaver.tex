%-----------------------%
%  KAPITOLA 00 - ZÁVĚR  %
%-----------------------%

\chapter*{ZÁVĚR}
\phantomsection
\addcontentsline{toc}{chapter}{ZÁVĚR}

Oblíbenost systému \LaTeX\ společně s jeho nesporně vysokou kvalitou typografické sazby dokumentů dala vzniknout šabloně závěrečných prací pro Fakultu strojního inženýrství na Vysokém učení technickém v Brně. Šablona je navržena s ohledem na kvalitativní požadavky vysokoškolských kvalifikačních prací a zároveň respektuje pokyny stanovené Směrnicí děkana č. 3/2022, která je doplňkem Směrnice rektora č. 72/2017 "Úprava, odevzdávání a zveřejňování závěrečných prací".

Šablona byla vytvořena s ohledem na jednoduchost a obsahuje podrobné popisy používaných příkazů včetně příkladů použití v textu práce i přímo ve zdrojovém kódu.

Jádrem celé šablony jsou tři pracovní soubory -- nastavení, hlavní pracovní soubor a stylový balíček.

Soubor nastavení poskytuje prostor s vlastními příkazy pro vyplnění základních údajů o závěrečné práci, jako jsou název práce, její typ, jméno autora a vedoucího, jazyk zpracování a další. Zároveň nabízí možnost volby druhu oddělování odstavců a typu číslování obrázků, tabulek a rovnic.

Hlavní pracovní soubor je soubor, který řídí tvorbu celého dokumentu. Uživatel v něm aktivuje samotnou šablonu, vepisuje úvodní textové náležitosti (abstrakt, klíčová slova, bibliografickou citaci a další), importuje titulní stranu, zadání a jednotlivé kapitoly.

Stylový balíček je zdrojový soubor šablony, který obsahuje definici všech vlastních příkazů vytvořených pro potřeby šablony. Příkazy upravují formální i funkční stránku šablony. K tomu jsou využívány i externí balíčky, které rozšiřují základní funkcionality systému \LaTeX.

Pro přehlednost jsou v adresáři se šablonou vytvořeny podadresáře pro třídění jednotlivých souborů (kapitoly, přílohy, obrázky, písma a další), které zpřehledňují práci se šablonou.

Tato závěrečná práce obsahuje také úvodní rešeršní kapitolu s krátkým přehledem nejpoužívanějších \LaTeX\ editorů s jejich popisem a shrnutím výhod a nevýhod včetně doporučení vhodných editorů pro široké využití.

V části vlastního řešení se následně jednotlivé kapitoly věnují popisu všech tří pracovních souborů, kde jsou ukázány všechny vytvořené příkazy, jak vypadají, jak fungují, jaké mají přednastavené hodnoty a jak se příkazy zachovají, pokud budou uživatelem nesprávně použity.

Šablona poskytuje jednoduchý nástroj pro tvorbu závěrečných prací s potenciálem stát se nejenom jednotnou fakultní šablonou, ale také jednotnou šablonou pro celé Vysoké učení technické v Brně.

%-----------------------%